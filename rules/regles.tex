\documentclass[a4paper, 12pt]{article}
\usepackage{tbagrelstandard}
\usepackage{charter}
\setlength{\parskip}{0.5\baselineskip}

\begin{document}

    \fancytitle{Consignes sur les \sc{tipe}}{0.45\linewidth}{0.25cm}{\LARGE\bfseries}

    % \ornament{88}{10cm}

    \section{Pour faire un bon oral\ldots}

    \begin{itemize}
        \item être capable de parler distinctement ;
        \item avoir au moins 10 min de choses intéressantes à dire ;
        \item ne pas citer de noms, il n'y a pas de prérequis, il faut donc expliquer les notions pour que cela soit compréhensible même par des novices dans le domaine ;
        \item il faut que les premières phrases soient accrocheuses (mais aussi un sujet intéressant) pour étonner le jury, tout en étant le plus compréhensible possible !
        \item ça peut être un bonne idée de commencer par un exemple (ou un résultat paradoxal\ldots)
        \item[$\star$] dans la partie débat, il faut être capable de rapprocher immédiatement une notion annexe proposée par le jury à une notion du programme
        \item avoir des idées sur les ordres de grandeur de toutes les données
    \end{itemize}

    \section{Milieux en mathématiques}

    \begin{itemize}
        \item renvoie à l'espace dans lequel vivent les objets que l'on manipule ($\mcc{M}_n(\mathbb{K})$) ;
        \item peut s'enrichir de nombreuses structures : champ gravitationnel par exemple ;
        \item quelle est la durée nécessaire à deux masses pour se rejoindre dans le vide ?
        \item voir problème du triangle qui est posé dans $\R^2$ mais qui se résout de manière triviale dans $\R^3$ ie dans un espace de dimension supérieure.
    \end{itemize}

    \section{Idée de sujet : définir un centre-ville}

    Comment pourrait-on définir un centre-ville, d'un point de vue géométrique tout d'abord ?

    On peut penser à tous les points plus proches du centre du contour que de l'extérieur.

\end{document}
